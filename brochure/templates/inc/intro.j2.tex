\chapter{Informations}

\section*{Organisateurs}

Passeport vacances Fribourg, ch. du Grabensaal 4, 1700 Fribourg

\section*{Site Internet}

\url{www.pvfr.ch}

\section*{Facebook}

\url{https://www.facebook.com/passeportvacancesfribourg}

\section*{Adresse électronique du secrétariat}

\url{info@pvfr.ch}

\section*{Age}

Le passeport est destiné aux enfants et aux jeunes de 6 ans révolus au 31 août \VAR{meta.year} à 16 ans.

\newpage
\section*{Prix et durée du passeport}

Le prix d'un passeport est de 20 francs pour une semaine, 35 francs pour deux semaines, 50 francs pour trois semaines,
et de 60 francs pour quatre semaines (sauf dans les rares cas où la commune du détenteur ne participe pas
financièrement).

Il est valable entre le \VAR{meta.start.0} et le \VAR{meta.end.0} \VAR{meta.year} et permet de participer à la grande fête de clôture qui aura lieu
le \textbf{\VAR{meta.cloture.0} \VAR{meta.year}, même avec un passeport périmé}.

Le passeport du troisième enfant d'une même famille est gratuit, sur présentation du livret ou du certificat de famille.
Dans ce cas, les passeports doivent être achetés en même temps et pour le même nombre de semaines.

Le passeport est personnel et les enfants doivent l'avoir sur eux lors de chaque activité. En cas de perte ou s'il n'est
pas utilisé, il n'est pas remplacé, ni remboursé.

\section*{Prestations (v. \og{}Activités spontanées et à coupon\fg{})}

Dans le prix du passeport sont compris:

\begin{itemize}
\item la fête de clôture qui aura lieu le \VAR{meta.cloture.0} \VAR{meta.year}, même avec un passeport périmé;
\item un libre accès à l'ensemble du réseau régional et urbain des TPF\footnote{Dans la description des activités,
nous indiquons quel ligne de bus prendre et à quel arrêt descendre. Avec le numéro de la ligne, nous indiquons
également la direction \textbf{si tu prends le bus depuis la Gare}. Par exemle: \og{}ligne 4 Auge, arrêt: Palme\fg{} signifie que tu
dois prendre le bus de la \emph{ligne 4}; \textbf{si tu pars de la gare}, tu prends le bus direction \emph{Auge}; et tu
descends à l'arrêt \emph{Palme}.};
% A VOIR >
\item divers coupons à utiliser pour une partie gratuite de minigolf et de bowling ou pour se rendre dans des commerces
de la place qui vous offriront une viennoiserie, une boisson ou autres;
\item l'entrée gratuite dans plusieurs musées et aux Bains de la Motta;
\item l'accès gratuit à la Tour de la Cathédrale Saint-Nicolas;
\item le prêt de livres ou d'un jeu;
\item un tour en petit train du Gottéron;
\item du hockey et du patinage libre \VAR{meta.patinoire.0};
\item le concours \og{}Pinocchio\fg{};
\item l'accueil midi, tous les jours, du lundi au vendredi de 11h30 à 13h30, à la ferme du Grabensaal
(prendre un pique-nique avec boisson);
\item l'accueil midi \og{}Grillades\fg{}, tous les mardis à la ferme du Grabensaal de 11h00 à 13h00,
(prendre quelque chose à griller avec boisson).

\end{itemize}
\newpage
\section*{Déroulement}

\subsection*{Activités avec inscription}

\BLOCK{if opts.light}
Le délai d'inscription pour les activités avec inscription est dépassé et ces activités ne sont plus disponibles.
\BLOCK{else}
Le programme du Passeport vacances comprend également des activités qui nécessitent une inscription par Internet.
\textbf{Entre le \VAR{meta.inscription.start.0} et le \VAR{meta.inscription.end.0} \VAR{meta.year} à 16h00}, allez
sur le site www.pvfr.ch. Vous y trouverez le programme du
Passeport vacances \VAR{meta.year} au format PDF et la marche à suivre. Il est important que vous consultiez cette dernière, car
elle contient toute la démarche à effectuer pour mener à bien votre inscription et votre choix d'activités. En voici
un résumé:

\begin{enumerate}
\item Rendez-vous auprès d'un des quatre points de vente pour acheter votre passeport et remplir une fiche. A ce moment-là,
vous devrez donner votre \textbf{adresse e-mail} et une \textbf{photo au format passeport}. Vous recevrez votre
passeport et le programme \VAR{meta.year} en format papier.
\item Allez sur le site \url{www.pvfr.ch} et cliquez sur \og{}Connexion\fg{}. Utilisez le nom d'utilisateur et le mot de passe
qui se trouve au dos de votre passeport. Autrement vous pouvez aussi cliquer sur l'e-mail que vous avez dû reçevoir.
Si vous n'avez pas reçu d'e-mail de Groople, contactez-nous.
\item Si vous ne souhaitez pas participer à des activités certains jours, \textbf{décochez} ces jours (les cases
deviennent rouges). Les cases des jours où vous êtes disponibles pour des activités restent cochées en vert.
\item Sur la page suivante, vous voyez toutes les activités en rapport à votre âge et vos disponibilités. Pour avoir
plus d'informations sur une activité, posez la pointe de votre curseur sur le point d'interrogation. Choisissez
maintenant vos activités \textbf{par ordre de préférence}, en cliquant sur les bulles de couleur.
\item Au bas de la page, vous voyez votre choix d'activités dans l'ordre de vos préférences. Vous pouvez les enlever en
cliquant sur les croix rouges.
\item Dans la dernière page s'affiche un résumé de vos choix. Si vous êtes satisfait-e, cliquez sur \og{}Terminer\fg{}.
\item Vous allez recevoir un e-mail \textbf{de Groople} avec la confirmation que vos choix ont bien été enregistrés et
un récapitulatif. \textbf{Vous pouvez encore modifier vos préférences jusqu'au \VAR{meta.inscription.end.0} à 16h00};
\item A la fin juin, le programme informatique attribuera les activités d'une manière équitable, puis vous recevrez un
e-mail vous informant qu'une bourse aux places restantes sera mise en ligne;
\item Au plus tard \textbf{le \VAR{meta.final_program.0}}, vous recevrez par courriel la liste définitive des activités auxquelles vous
participerez.
\end{enumerate}
\BLOCK{endif}

\BLOCK{if opts.light}
\section*{Point de vente des passeports}
\BLOCK{else}
\section*{Points de vente des passeports}
\BLOCK{endif}

\begin{itemize}
\item Office du tourisme de Fribourg, place Jean-Tinguely 1 (bâtiment Equilibre)
\BLOCK{if not opts.light}
\item Secrétariat communal de Givisiez
\item Secrétariat communal de Marly
\item Secrétariat communal de Villars-sur-Glâne
\BLOCK{endif}

\end{itemize}

\BLOCK{if not opts.light}
\newpage
\section*{Inscriptions assistées}

Si vous n'avez pas la possibilité de vous inscrire sur Internet ou si vous rencontrez des difficultés, nous vous donnons
rendez-vous à l'Office du tourisme de Fribourg, place Jean-Tinguely 1 à Fribourg (bâtiment Equilibre)
\textbf{le \VAR{meta.equilibre.0}}. Des membres du comité se tiendront à votre disposition pour vous aider dans
le processus d'inscription et répondre à vos questions.
\BLOCK{endif}

\section*{Pour les activités ayant lieu dans les endroits suivants, voici comment vous y rendre:}

\subsection*{La ferme du Grabensaal}

Prenez le bus TPF ligne 4 Auge, arrêt \og{}Auge, Sous-Pont\fg{}. Puis, en face de l'école des Neigles, traversez la passerelle
qui mène à la ferme.

\subsection*{Ch. du Gottéron}

Les locaux du Gottéron se trouvent au ch. du Gottéron 15 et 17. Prenez le bus TPF ligne 4 Auge, jusqu'à l'arrêt
\og{}Palme\fg{}. Dirigez-vous ensuite vers la vallée du Gottéron et marchez environ 7 minutes.

\section*{Horaires}

Nous demandons aux participants d'être à l'heure. Les organisateurs ou les accompagnants ne sont pas tenus d'attendre.
Lorsque le bus est utilisé pour se rendre à une activité, il se peut que l'heure du retour ne soit pas tout à fait
respectée, en raison du trafic sur les routes. Nous vous remercions d'avance pour votre compréhension.

\section*{Discipline}

Les personnes inscrites à une activité s'engagent à y participer. Après chaque activité, les jeunes remercient
l'organisateur.

\section*{Assurances}

Tous les participants doivent être assurés en cas d'accident et sont responsables des dommages qu'ils pourraient causer.
Les organisateurs déclinent toute responsabilité en cas d'accident.

\section*{Droit à l'image}

Pendant les activités, il se peut que les enfants soient photographiés et filmés par les organisateurs, des journalistes
ou par des membres du comité d'organisation. Si l'enfant ne souhaite pas apparaître sur une photo ou sur une vidéo,
il peut simplement en informer l'organisateur.

En cochant la case correspondante sur le passeport,
vous donnez expressément votre consentement à ce que votre fille ou votre fils soit photographié et filmé dans le cadre du
Passeport vacances Fribourg et à ce que ces images puissent être publiées sur le site Internet du Passeport vacances Fribourg
ainsi que sur les réseaux sociaux et d’éventuelles brochures.

Le Passeport vacances Fribourg s’engage à trier les images soumises à publication, dans le respect des enfants.
Notez que vous pouvez en tout temps exiger le retrait d’une image si sa publication ne vous convient pas.

\section*{Comment nous soutenir ?}

\begin{description}
	\item [Par un don]
	Chaque don sera apprécié à sa juste valeur (BCF / CH74 0076 8300 1516 4600 9)
	\item [En nous proposant du matériel]
	Nous avons toujours besoin de matériel divers pour réaliser et préparer
	nos activités. Vous pouvez nous faire des propositions en nous adressant
	un courriel à \url{info@pvfr.ch}
	\item [Par du bénévolat]
	Si vous souhaitez participer bénévolement à nos activités pour les
	enfants, n’hésitez pas à nous envoyer un courriel à \url{info@pvfr.ch}
\end{description}

\textbf{Les organisateurs remercient toutes les personnes qui, par leur aide bénévole et leur disponibilité, permettent
la mise sur pied du Passeport vacances. Un merci tout particulier à tous les généreux donateurs pour leur appui financier.
}
